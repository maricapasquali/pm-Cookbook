\documentclass{article}
\usepackage{graphicx} % Required for inserting images
\usepackage[italian]{babel}
\usepackage[hidelinks]{hyperref}
\usepackage{todonotes}
\usepackage{biblatex}
\usepackage{listings}
\usepackage{parskip}

\title{
    Project Management \\
    \textbf{ 
        User Stories for\\ 
        Web Application: \\
        \textit{Google CookBook}
    }
}
\author{
    Marica Pasquali \\ 
    (\href{mailto:marica.pasquali@studio.unibo.it}{marica.pasquali@studio.unibo.it})
}

\begin{document}

\maketitle
\newpage
\tableofcontents
\newpage

\section{Servizio di autenticatione}
\subsection{Registrazione}
\begin{itemize}
    \item Come Utente Anonimo, voglio iscrivermi alla piattaforma in modo da sfruttare tutte le funzionalità 
    che la piattaforma offre
\end{itemize}

\subsection{Login}
\begin{itemize}
    \item Come Utente (non anonimo), voglio effettuare il log-in alla piattaforma con username e password 
    in modo da poter accedere alle mie informazioni personali e sfruttare tutte le funzionalità che 
    la piattaforma offre
\end{itemize}

\subsection{Gestione degli account utente}
\begin{itemize}
    \item Come Utente Registato, voglio poter ricevere notifiche nel caso vengono modificate le mie 
    informazioni personali in modo tale da sapere se qualcuno abbia violato il mio account
    \item Come Utente Registato, voglio poter cancellare il mio account perchè non mi interessa più 
    prendere parte alla community
    \item Come Utente Amministratore, voglio poter bandire un Utente Registrato (ossia cancellare il suo account)
    perchè ha violato le condizioni di utilizzo della piattaforma.
\end{itemize}


\section{Servizio di gestione delle ricette}
\subsection{Gestione degli alimenti/ingredienti}
\begin{itemize}
    \item Come Utente Amministratore o Utente Registrato, voglio inserire un nuovo alimento nella piattaforma 
    così che un altro utente non debba farlo
    \item Come Utente Registrato, voglio poter gestire una lista della spesa per poter ricordare di comprare 
    gli alimenti che mi servono per cucinare
\end{itemize}

\subsection{Gestione delle ricette}
\begin{itemize}
    \item Come Utente Generico, voglio `sfogliare' il catalogo delle ricette così da poter sceglierne una da 
    cucinare
    \item Come Utente Generico, voglio cercare una ricetta senza latticini così da poter cucinare per la mia 
    ragazza senza che stia male perchè intollerante al lattosio
    \item Come Utente Generico, voglio cercare una ricetta coreana così da impressionare i miei amici con una
    ricetta esotica
    \item Come Utente Generico, voglio cercare una ricetta contenente il tofu per poterlo consumare perchè è 
    in prossimita alla scadenza
    \item Come Utente Generico, voglio cercare una ricetta per fare un dolce per il compleanno di mio figlio 
    \item Come Utente Registrato, voglio poter postare una mia ricetta così da farla conoscere alla community 
    \item Come Utente Registrato, voglio poter salvare delle ricette senza però condividerle così da migliorarle 
    e postarle successivamente
    \item Come Utente Registato, voglio poter ricevere notifiche nelle nuove ricette degli altri utenti 
    così da poter essere sempre aggiornato sulle ultime novità
\end{itemize}

\subsection{Gestione dei giudizi/commenti alle ricette postate}
\begin{itemize}
    \item Come Utente Generico, voglio poter mette un like o scrivere un commento a una ricetta postata per 
    esprimere la mia opinione e/o per dare dei suggerimenti
    \item Come Utente Registato, voglio poter ricevere notifiche nel caso un utente commentasse un mio commento 
    così da poter essere sempre aggiornato sulle ultime novità
    \item Come Utente Amministratore, voglio poter cancellare un commento segnalato da un utente generico perchè 
    il suo contenuto non è appropriato 
\end{itemize}

\section{Servizio di gestione delle comunicazioni}
\subsection{Gestione delle amicizie tra gli utenti}
\begin{itemize}
    \item Come Utente Registrato, voglio chiedere l'amicizia a un utente registrato così da creare una cerchia 
    più ristretta della community
    \item Come Utente Registrato, voglio poter ricevere notifiche sulle richiede di amicizia così da essere 
    informato subito
\end{itemize}

\subsection{Gestione delle chat tra utenti}
\begin{itemize}
    \item Come Utente Registrato, voglio contattare in modo privato delle mie amiche così da poter discutere di 
    cucina con loro senza discutere gli altri utenti
    \item Come Utente Registrato, voglio contattare l'amministatore della piattaforma perché ho riscontato un 
    problema nella creazione di una ricetta
    \item Come Utente Registrato, voglio scambiare le mie ricette (anche quelle non pubblicate) con le mie amiche 
    così da poter ampliare e migliorare la mia conoscienza della cucina    
\end{itemize}

\newpage

\section*{Glossario}

\textit{\textbf{Utente anonimo}}: è un utente che non è registrato alla piattaforma

\textit{\textbf{Utente registrato}}: è un utente che ha effettuato la registrazione alla piattaforma

\textit{\textbf{Utente generico}}: è un utente anonimo oppure un utente registrato

\textit{\textbf{Utente amministratore}}: è un utente che gestisce l'amministrazione della piattaforma e quindi a privilegi di amministratore

\end{document}