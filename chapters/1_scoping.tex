\chapter{Scoping}

La fase di scoping nasce dopo che ricerche sulla potenzialità di mercato della piattaforma e sul panorama
competitivo hanno dato esiti positivi; questo porta alla conduzione di uno scoping meeting, in più sessioni, 
orientato alla produzione di un Project Overview Statement (POS) e alla raccolta dei requisiti.
Allo scoping meeting partecipano diverse figure chiave per la realizzazione del progetto:
\begin{itemize}
    \item Project Managers
    \item Sponsors
    \item Facilitatori
    \item Development team
    \item Tecnografo
\end{itemize}

\section{Project Scoping Meeting}

\subsection*{Primo meeting}
\subsubsection*{Ouput}
\begin{itemize}
    \item Condition of Satisfaction (CoS)
\end{itemize} 
\subsubsection*{Resoconto}
Gli sponsors del progetto illustano in maniera dettagliata ai partecipanti l'idea del progetto, con spiegazione 
conprovata dai dati e dalle analisi condotte durante le ricerche di mercato preliminari allo scoping meeting. 
Fornita la descrizione del background, si parte con una definizione più chiara e precisa dei desiderata e delle caratteristiche,
anche dal punto di vista tecnologico, che la piattaforma deve presentare, allo scopo di permettere al development team di studiare 
la situazione e di fare una prima proposta per una strategia realizzativa della soluzione.        
\subsubsection*{Agenda successiva}
\begin{itemize}
    \item esaminare le proposte del development team
    \item definizione del modello di business
\end{itemize} 



\subsection*{Secondo meeting}
\subsubsection*{Ouput}
\begin{itemize}
    \item Modello di Business
\end{itemize} 
\subsubsection*{Resoconto}
E' stato effettuato un rapido riassunto di quanto emerso nel primo scoping meeting. In questa seconda iterazione 
si è effettuato la presentazione da parte del development team della strategia e degli strumenti 
che saranno necessari alla realizzazione della soluzione, la quale a permesso di dirigere il modello di business.
\subsubsection*{Agenda successiva}
\begin{itemize}
    \item inizio della raccolta dei requisiti
\end{itemize} 



\subsection*{Terzo meeting}
\subsubsection*{Ouput}
\begin{itemize}
    \item bozze dei requisiti
\end{itemize} 
\subsubsection*{Resoconto}
E' stato effettuato un riassunto di quanto emerso dalle riunioni precedenti.
Questa terza iterazione è interamente focalizzata sulla stesura formale dei requisiti del prodotto attraverso l'utilizzo di mockups e story board.
\subsubsection*{Agenda successiva}
\begin{itemize}
    \item continuazione della raccolta dei requisiti
\end{itemize} 



\subsection*{Quarto meeting}
\subsubsection*{Ouput}
\begin{itemize}
    \item versione finale dei requisiti
\end{itemize} 
\subsubsection*{Resoconto}
E' stato effettuato un riassunto di quanto emerso dalle riunioni precedenti.
Questa quanta iterazione è interamente focalizzata sulla stesura formale dei requisiti del prodotto, 
che viene completata entro il termine di questa quanta fase.  
\subsubsection*{Agenda successiva}
\begin{itemize}
    \item definizione del modello PMLC
    \item definizone del POS
\end{itemize} 



\subsection*{Quinto meeting}
\subsubsection*{Ouput}
\begin{itemize}
    \item POS
\end{itemize} 
\subsubsection*{Resoconto}
E' stato effettuato un riassunto di quanto emerso dalle riunioni precedenti.
Questa ultima sessione porta alla definizione del POS e alla scelta del modello PMLC più adeguato. 
L'approvazione del POS da parte degli sponsors e dal senior manager porta il progetto nella fase di planning.


\section{Documenti redatti}

\subsection{Condition of Satisfaction (CoS)}

Le Condtions of Satisfaction (CoS) sono quelle `condizioni' che devono essere rispettate 
per garantire il successo del progetto. CoS giudano la definizione dei requisiti e il processo decisionale
durante tutto il ciclo di vita del progetto.

Le CoS emerse sono disponibile nei documenti in allegato.

\subsection{Business Model Canvas}
Il Business Model Canvas è un potente framework all'interno del quale sono rappresentati, sotto
forma di blocchi, nove elementi che definiscono il modo con cui \textit{CookBook} intende creare, distribuire
e raccogliere valore.

Il Business Model Canvas redatto è disponibile nei documenti in allegato.

\subsection{Requisiti}

L'approccio utilizzato per la raccolta dei requisiti è \textit{Facilitated Group Session}, ossia riunioni più piccole in cui solo alcuni tipi di utenti 
vengono interpellati perchè serve fare domande rapide e mirate.

Una conoscenza pregressa delle funzionalità principali previste dagli applicativi gestionali e delle caratteristiche delle piattaforme social porta ad 
ipotizzare quali potrebbero essere le feature che un utente, che sia uno amatoriale o un professionista, si aspetterebbe da \textit{CookBook}.
Tuttavia, si sceglie di ricorrere ad un consulente super-partes imparziale, un professionista che fornisca la sua esperienza per supportare questa fase 
embrionale dello sviluppo dell'attività e facilitare il processo di requirements gathering: l'obiettivo è dettagliare immediatamente, documentare e verificare le caratteristiche.

Per la raccolta dei requisiti funzionali sono stati utilizzati dei mockups e delle storyboard per facilitare l'interazione con i facilitari.

I documenti riguardanti i requisiti sono disponibili nei documenti in allegato.

\subsection{Project Overview Statement (POS)}

Project Overview Statement (POS) è un documento che esprime una descrizione sintetica del progetto.
Solo se il POS è approvato dal Senior Manager, si ha il nulla osta a procedere con la fase di planning.

Il POS redatto è disponibile nei documenti in allegato.


\section{Modello PMLC}
\todo{scelta del modello PMLC}