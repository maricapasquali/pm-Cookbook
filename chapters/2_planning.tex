\chapter{Planning}
Durante la fase di Planning vengono identificate le attività che devono essere svolte per 
implementare i requisiti e completare il progetto. Inoltre vengono stimati i tempi e le 
risorse necessarie. 
Tutto questo avviene in delle riunioni che in questa fase sono denominate, Joint Project 
Planning Sessions (JPPS).
Si è prevista una durata massima di quattro giorni. Potrà essere fatta un'eccezione per
l'aggiunta di un giorno nel caso in cui si riscontrino particolari difficoltà.

\section{Joint Project Planning Sessions}
Alle Joint Project Planning Sessions partecipano diverse figure chiave per la realizzazione 
del progetto:
\begin{itemize}
    \item Project Managers
    \item Sponsors
    \item Facilitatori
    \item Core Team
    \item Tecnografo
\end{itemize}

Inoltre sono state identificate anche le facilities e l'equipaggiamento utilizzate nelle JPPS:
\begin{itemize}
    \item Sala riunioni privata
    \item Proiettore e PC 
    \item Lavagna e pennarelli
    \item Post-it colorati
\end{itemize}

Di seguito viene ripotata l'agenda delle JPPS, le quali si dividono in due fasi:
\begin{itemize}
    \item \textit{Kick-off} meeting che si articola in:
    \begin{itemize}
        \item introduzione degli sponsors e della loro importanza per il progetto
        \item introduzione dei partecipanti e del ruolo che rivestono nel progetto e per il 
        meeting stesso
    \end{itemize}
    \item \textit{Working Session} che comprende:
    \begin{itemize}
        \item la scelta dell'approccio necessario per la pianificazione del progetto: 
        \textit{Scrum}
        \item la validazione e la prioritizzazione delle user stories, mediante metodo \textit{MoSCoW} 
        e generazione del Product Backlog
        \item stima della durata del progetto in termini di sprint prendenendo in considerazione 
        la stima della durata delle attività e delle risorse definite
        \item identificazione dei rischi e degli eventuali piani di mitigazione
        \item ottenere il consenso di tutti i partecipanti sui contenuti del piano
    \end{itemize}
\end{itemize}


\section{Documenti redatti}

\subsection{Approccio di pianificazione}
La descrizione dell'approccio di pianificazione è disponibile nei documenti in allegato.

\subsection{Risorse necessarie}
Le stime delle risorse necessarie sono disponibili nei documenti in allegato.

\subsection{Analisi dei rischi}
L'analisi dei rischi è disponibili nei documenti in allegato.
