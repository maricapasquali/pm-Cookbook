\chapter*{Introduzione}
\addcontentsline{toc}{chapter}{Introduzione}

Tra il 2020 e 2021, il motore di ricerca più popolare al mondo, Google, 
ha notato un incremento delle ricerche riguardanti il cibo da preparare in casa; 
questo può essere dovuto al lockdown per la pandemia Covid in cui i 
ristoranti/fastfood erano chiusi e quindi le persone, per mangiare, 
dovevano `mettersi ai fornelli'. 
\footnote{Affermazione fittizia per avere un background da cui partire}

Con questo incremento, Google ha avuto l'idea di creare una piattaforma social denominata 
\textit{Google CookBook}, per la condivisione di ricette culinarie con l'intento di 
creare un unico ecosistema attraverso il quale ricercare, promuovere e sviluppare 
la conoscenza e la pratica delle arti culinarie.

L'obiettivo è costruire una rete che raggiunga tutti gli appassionati, ma anche i 
professionisti, della cucina, accorciando le distanze tra le persone, favorendo la 
creazione di una community attiva e dinamica.
