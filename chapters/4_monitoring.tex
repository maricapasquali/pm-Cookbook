\chapter{Monitoring}

La fase di Monitoring consiste nei processi necessari per seguire, revisionare e regolare l'avanzamento
e le prestazioni del progetto, identificare le eventuali aree in cui sono necessarie modifiche al piano e avviare le 
relative modifiche.

Il monitoraggio continuo fornisce al gruppo di progetto e agli sponsors una visione approfondita dello stato del progetto
e identifica le eventuali aree che richiedono maggiore attenzione.

\section{Project Status Meetings}

\subsection*{Daily Scrum}

Ogni giorno vengono organizzate le Daily Scrum, riunioni gionaliere in cui ogni membro del team riporta brevemente 
lo stato del proprio lavoro. 
In particolare, lo stato può essere:
\begin{itemize}
    \item in schedula
    \item in anticipo rispetto a quanto previsto
    \item in ritardo rispetto a quanto previsto. Sarà necessario specificare di quanto e se c'è bisogno di un aiuto.
\end{itemize}

\subsection*{Sprint Review}

Ogni due settimane viene organizzata la Sprint Review, riunione con gli sponsors per mostrare i progressi. 
In questo modo si hanno feedback continui e si può correggere eventuali problemi in modo tempestivo. 

La riunione è strutturata secondo la seguente agenda:
\begin{itemize}
    \item Introduzione: viene descritto il lavoro fatto nelle ultime due settimane, ponendo particolare enfasi sul valore aggiunto dalle nuove feature
    \item Demo del sistema: viene mostrato brevemente il sistema in funzione e come l'utente può interagire con esso
    \item Discussione: gli sponsors possono fare domande e richiedere chiarimenti sulle nuove funzionalità, inoltre esprime i feedback su quanto visto
\end{itemize}

\subsection*{Sprint Retrospective}

Dopo la Sprint Review viene organizzata una Sprint Retrospective, riunione in cui i membri dei team identificano i processi 
che hanno funzionato e non hanno funzionato durante lo sprint. 
Questo incontro è focalizzato sui miglioramenti dei processi per ottimizzare il flusso di lavoro dello sprint del tuo team.

La riunione è strutturata secondo la seguente agenda:
\begin{itemize}
    \item Stabilire l'obiettivo della riunione
    \item Raccogliere i feedback del team
    \item Impostare le azioni, ossia creare elementi di azione per affrontare i punti critici comuni
\end{itemize}


\section{Sistemi di reporting}

Il Team mantiene \textbf{Issues Log}, registri che hanno la finalià di tracciare tutte le problematiche emerse durante 
lo svolgimento dei lavori. Gli Issues Log vengono tenuti per l'ambito riguardante il marketing e la gestione del progetto, 
mentre per l'ambito tecnologico e di sviluppo viene utilizzata la \textbf{Scrum Board}, una rappresentazione visiva del lavoro 
che un team Scrum deve svolgere. 
La Scrum Board consiste effettivamente in una lavagna in cui vengono segnati i progressi dello Sprint, dal principio, fino al 
completamento. Infatti, tramite la Scrum Board è possibile isolare e organizzare i diversi compiti del gruppo di lavoro, così 
come monitorare ogni singolo task tramite il suo ciclo vita.


Il Team redige sia i \textbf{Current Period Reports}, i quali evidenziano le attività completate più rilevanti e le eventuali 
variazioni rispetto a quanto pianificato durante lo Sprint, sia \textbf{Cumulative Reports}, i quali sono efficaci nel mostrare
i trend nell'avanzamento del progetto.
