\chapter{Launching}
La fase di launching è la terza fase del ciclo di vita del project management 
e solitamente è la fase più lunga del progetto. 
Durante questa fase, il team di progetto sviluppa il prodotto e presenta il prodotto 
finale agli Sponsors.



\section{Recruting del Project Team}

\begin{itemize}
    \item \textbf{Core Team}: team composto da 2 sviluppatori senior
    \item \textbf{Project Manager}: ruolo ricoperto dallo stesso manager che ha
    presenziato durante le riunioni di scoping e planning
    \item \textbf{Product Owner}: membro del core team che affianca il Project Manager
    \item \textbf{Developer Team}: 4 sviluppatori junior e un database administrator
    \item \textbf{Client Team}: membri identificati dagli Sponsors, questi forniscono feedback
    sui risultati ottenuti ad ogni sprint in modo da indenficare quanto prima eventuali problemi
\end{itemize}

\section{Kick-off meeting}
Questa fase prevede una riunione iniziale, denominata \textit{Kick-off meeting} nella
quale viene annunciato che il progetto pianificato è stato approvato e vengono essenzialmente 
presentati tutti i partecipanti.

Di seguito viene riportato una sintesi del Kick Off meeting.

\textbf{Agenda}

E' stato presentato, dal project manager, il progetto specificando lo scopo, il business value e 
la soluzione che si intende sviluppare. Ogni membro del team di sviluppo è stato presentando, 
specificando per ognuno il suo ruolo e le sue responsabilità all'interno del progetto.
Il project manager ha presentato la documentazione che è stata redatta nelle fasi precedenti.
Sono state discusse le regole operative ed il piano per la qualità tecnica che il team deve seguire 
durante lo svolgimento del progetto.

\textbf{Output}:
\begin{itemize}
    \item Regole operative per il team
\end{itemize}

\section{Documenti redatti}

\subsection{Regole operative per il team}
Stabilire le regole operative per il team di progetto è fondamentale per concordare i processi che devono essere
eseguiti.

Le regole operative per il team sono disponibili nei documenti in allegato.
