\chapter{Closing}

La fase di Closing è la fase in cui vengono eseguiti i processi per completare o concludere formalmente un progetto.
Questa fase verifica che i processi previsti nelle fasi precedenti siano completati al fine di chiudere il progetto.
I principali vantaggi di questa fase è che le fasi, i progetti e i contratti sono chiusi nel modo corretto.

\section{Installazione del software}

Essendo un'applicazione ``nuova'' (ossia progettata e sviluppata da zero), si è deciso di utilizzare un 
\textbf{approccio phased} per il deploy in Cloud del software. Questo approccio prevede la decomposizione 
del deliverable in tante parti che sono state sviluppate e consegnate in sequenza durante i vari Sprint. 

\section{Documentazione di progetto}

In allegato, è possibile consultare la documentazione di progetto prodotta, un asset importante per 
\textit{Google Cookbook} e per il suo successo.

\section{Audit post-implementazione}

\textbf{Partecipanti}
\begin{itemize}
    \item Project Managers
    \item Programmatori (senior e junior)
    \item Sponsors
\end{itemize}

\textbf{Resoconto}
\begin{itemize}
    \item \textit{Obiettivi}: il sistema risulta funzionante nella sua interezza e gli obiettivi del progetto sono 
    stati raggiunti
    \item \textit{Risorse}: le stime effettuate sono risultate corrette, permettendo la riuscita del progetto nei
    tempi e costi
    \item \textit{Specifiche}: tutte le specifiche sono state rispettate
    \item \textit{Deliverables}: i deliverables di progetto riescono a raggiungere il loro scopo e sono stati testati 
    in modo da garantirne la correttezza, inoltre, anche gli sponsors hanno confermato il corretto funzionamento 
    del sistema
    \item \textit{Business value}: gli sponsors hanno confermato il valore apportato dal nuovo software
    \item \textit{Metodologia di gestione del progetto}: inizialmente i programmatori junior hanno avuto qualche 
    difficoltà ad adattarsi alla metodologia scelta e ai vincoli di qualità ed eccellenza tecnica richiesti, 
    queste sono state superate grazie all'aiuto dei programmatori senior e del project manager
\end{itemize}